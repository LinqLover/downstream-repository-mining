\section{Framework}
\label{sec:framework}

To tackle the problems stated in \cref{sec:introduction}, our goal is to build a system that allows package developers to survey the API usage of downstream dependencies for a chosen target package.
Concretely, we have identified three common questions of package developers about the usage of their packages:

\begin{enumerate}[label=Q\arabic*]
	\item \label[question]{q1} Which dependencies are using the target package, and what problems do they attempt to solve with it?
	\item \label[question]{q2} By how many dependencies is a particular package member being used?
	\item \label[question]{q3} In which contexts and constellations is a particular package member being used?
\end{enumerate}

Furthermore, we pose two organizational requirements to the tool solution to ensure its practical applicability for the target user group:

\begin{enumerate}[label=R\arabic*]
	\item \label[requirement]{req1} The tool runs out-of-the-box with a small footprint, without the user needing to perform any sophisticated setup steps or providing additional system resources for it to work.
	\item \label[requirement]{req2} The tool blends in with the usual workflow of package developers at the best possible rate.
\end{enumerate}

In essence, our proposed approach to collect the required data while fulfilling these requirements consists of three steps:
\begin{enumerate*}[label=(\roman*)]
	\item collect downstream dependency repositories,
	\item mine package usage samples from them,
	and \item aggregate and present these usage data to the user.
\end{enumerate*}
In the following sections, we will describe each step.
